\section{Conclusions} % max 1/2 page
% Obj. 1: Summariza what you did in the project
% Obj. 2: Lessons learned
\label{sec:conclusions}

In the end, 8 different Rayleigh channel simulators have been implemented and both their statistical and computational performance have been tested. Excluding \textit{Jakes'} and \textit{PopBeaulieu}'s simulators, all the others have very similar performance, and are usually very close to the ideal case. Different simulators shined for particular reasons: while \textit{Komninakis} has outstanding performance for what concerns both computational complexity and fading statistics, it also has quite poor correlation statistics; on the other hand \textit{Clarke}'s original model actually yields very good performance all around without resulting as slow as it may have been thought. One thing to say, though, is that while for all the other \textit{SOS} simulators further optimization could (possibly) be done, it would be much more difficult for the latter since no symmetries whatsoever are present.

Thanks to this project I had the possibility to try out many new MATLAB's programming techniques that I recently learned from a few Mathwork's courses. Now my code performs better, is more robust, and also well commented, documented and organized. Furthermore, I now have many simulators ready to use if ever needed and a full performance characterization for all of them.