\section{Introduction} % max 1 page
\label{sec:introduction}

Since the Seventies, simulating a good wireless channel was recognized to be fundamental for wireless systems design. It was acknowledged right from the start for being simpler, cheaper and more repeatable than a channel sounding: no external expensive gear was required and a new simulation could be done in a matter of seconds or less.

Early on, a simple yet fairly realistic channel model was presented by Clarke \cite{clarke}, on which all the subsequent simulators were based on and many different statistics were extrapolated. A few years later, Jakes created the de facto standard simulator \cite{jakes} for many years to follow. His proposal was to artificially impose certain simmetries to how the channel behaves, thus being able to simulate an approximation of Clarke's model but only with a fouth of the oscillators. Jakes' approach, though, had some problems and throughout the years many proposals (\cite{A1},\cite{A2},\cite{A3},\cite{C1},\cite{C2}) have been done in order to fix most, if not all of them.

Finally, while proposing a new simulator, \textit{Xiao, Zheng} and \textit{Beaulieu} \cite{B1} recognized how well \textit{Clarke}'s model actually beahves even with a low number of sinusoidal components. In fact, they proved how fast it converges to the ideal formulas, thus not needing as much oscillators as \textit{Jakes} (and all those that followed) actually thought.

The report is structured as follow: in Section \ref{sec:technical_approach} I will present the technical aspects of the project, starting from Subsection \ref{subsec:objectives} which delineates the main objectives, then in Subsection \ref{subsec:math_models} the mathematical models of all the implemented simulators will be carefully described and in Subsection \ref{subsec:scenario} an outline of the code structure will be presented. Finally, in Subsection \ref{subsec:complications} I will briefly talk about a few complications encountered while completing this project. In Section \ref{sec:results}, then, the results will be presented and lastly in Section \ref{sec:conclusions} the conclusions will be drawn.